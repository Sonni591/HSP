\section{Einführung}
\label{ch:einfuehrung}
Das vorliegende Dokument beschreibt die Handhabung und Funktionsweise der Software \gls{SMPLSP} zur Losgrößen- und Ressourceneinsatzplanung bei Fließproduktion. \\
Das Tool wurde im Rahmen des Projektstudium 1 (Hauptseminar 1) im Masterstudiengang Informatik bei Herrn Professor Dr.-Ing. Frank Herrmann entwickelt und soll den Studenten
eine Möglichkeit bieten, das im Buch „Produktion und Logistik“ von Günther und Tempelmeier \cite{Templ09} gezeigte Verfahren zur Losgrößen- und Ressourceneinsatzplanung bei Fließproduktion nachzuvollziehen.
Umgesetzt wurden sowohl das klassische Losgrößenverfahren, als auch die Losgrößenberechnung bei Mehrproduktproduktion.

\subsection{Klassisches Losgrößenverfahren}
Das klassische Losgrößenverfahren betrachtet die Herstellung mehrerer Produkte auf einer
einzelnen Produktionsanlage. Da die Berechnung der optimalen Losgröße und des optimalen
Produktionszyklus für jedes Produkt isoliert durchgeführt werden kann es zu einer zeitlichen Überschneidung der Produktionsaufträge kommen. Daher ist die Produktion mehrerer Produkte unter Einhaltung des berechneten Produktionsplans aus dem klassischen Losgrößenverfahren häufig nicht möglich. Eine Fallstudie zu diesem Verfahren ist in \cite{Templ09} verfügbar.

\subsection{Mehrproduktproduktion}
Das Problem der Überschneidung von Produktionsaufträgen bei der Mehrproduktproduktion auf einer Anlage lässt sich durch die Bestimmung eines für alle Produkte einheitlichen gemeinsamen Produktionszyklus vermeiden. Die Aufgabe dieses Verfahrens ist die Bestimmung eines solchen gemeinsamen Produktionszyklus, sowie die Berechnung der optimalen Losgröße je Produkt. Eine Fallstudie zu diesem Verfahren ist in \cite{Templ09} verfügbar.
\pagebreak